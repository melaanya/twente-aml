\begin{abstract}
Gradient boosting is one of the rapidly developing and widely used techniques which is acknowledged to be a powerful tool in both industrial problems and machine learning competitions. A recently appeared library CatBoost for gradient boosting on decision trees is claimed to outperform existing implementations by special processing of categorical variables and improved process of gradient estimation. It is claimed to outperform existing state-of-the-art models on several benchmarks for classification with no regards to any regression problems. This paper investigates the applicability of CatBoost  in the regression problem of housing price prediction. We compare its performance with the performance of two other gradient boosting libraries such as sklearn and XGBoost by training and evaluating them on three datasets with different properties. We also analyze categorical feature importance and their influence on performance.
\end{abstract}
