\documentclass[11pt,a4paper]{article}
\usepackage[utf8]{inputenc}
\usepackage[english]{babel}
\usepackage[left=2.5cm, right=2.5cm, top=2.5cm, bottom=2.5cm]{geometry}
\usepackage{hyperref}
\usepackage{graphicx}
\usepackage{times}
\usepackage[bottom]{footmisc}
\usepackage{listings}
\usepackage{bm}
\usepackage{amsmath}
\usepackage{algpseudocode}
\usepackage{algorithm}
\usepackage{titlesec}
\usepackage{booktabs}
\usepackage{amsmath}
\usepackage{xfrac}


\title{\textbf{Advanced Machine Learning -- Homework 2}}
\author{Anna Berger, Ashwin Sadananda Bhat (AML25)}


\begin{document}
	\maketitle
	
	\section*{Part 1}
	\subsection*{Exercise 1.1}
	
	In this part we consider the mixture of two Bernoulli distributions. To be more specific: each observation $x=(h, t)$ is the outcome of 5 coin tosses with a specific coin, with $h$ --- the number of heads and $t$ --- the number of tails, so $h+t=5$. But there are two coins in play with possible different probabilities for heads. Hence it can be modelled as a mixture of two probability distributions:
	$$ p(x | \mu,\pi) = \pi_1 \cdot p(x|\mu_1) + \pi_2 \cdot p(x|\mu_2) $$
	
	Each probability distribution is of Bernoulli type, so:
	$$ p((h,t)|\mu) = \binom {h+t}{h} \cdot \mu^h \cdot (1-\mu)^t$$
	
	We assume that parameters of the model are initialised as $\pi = \left(
	 \frac{1}{3}, \frac{2}{3}\right)$, $\mu = \left(
	 \frac{2}{3}, \frac{1}{2}\right)$.
	
	First, for each observation and for each component we compute responsibilities:
	
	$$ \gamma(z_{11}) = \frac{\pi_1 \cdot p(x_1|\mu_1)}{\pi_1 \cdot p(x_1|\mu_1)+[\pi_2 \cdot p(x_1|\mu_2)} = \frac{\binom {5}{2} \cdot \frac{1}{3} \cdot (\frac{2}{3})^2 \cdot (\frac{1}{3})^3}{\binom {5}{2} \cdot \frac{1}{3} \cdot (\frac{2}{3})^2 \cdot (\frac{1}{3})^3+\binom {5}{2} \cdot \frac{2}{3} \cdot (\frac{1}{2})^2 \cdot (\frac{1}{2})^3}=\frac{64}{307} $$
	
	$$ \gamma(z_{12}) = \frac{\pi_2 \cdot p(x_1|\mu_2)}{\pi_1 \cdot p(x_1|\mu_1)+\pi_2 \cdot p(x_1|\mu_2)} = \frac{\binom {5}{2} \cdot \frac{2}{3} \cdot (\frac{1}{2})^2 \cdot (\frac{1}{2})^3}{\binom {5}{2} \cdot \frac{1}{3} \cdot (\frac{2}{3})^2 \cdot (\frac{1}{3})^3+\binom {5}{2} \cdot \frac{2}{3} \cdot (\frac{1}{2})^2 \cdot (\frac{1}{2})^3}=\frac{243}{307} $$
	
	$$ \gamma(z_{21}) = \frac{\pi_1 \cdot p(x_2|\mu_1)}{\pi_1 \cdot p(x_2|\mu_1)+\pi_2 \cdot p(x_2|\mu_2)} = \frac{\binom {5}{1} \cdot \frac{1}{3} \cdot (\frac{2}{3})^1 \cdot (\frac{1}{3})^4}{\binom {5}{1} \cdot \frac{1}{3} \cdot (\frac{2}{3})^1 \cdot (\frac{1}{3})^4+\binom {5}{1} \cdot \frac{2}{3} \cdot (\frac{1}{2})^1 \cdot (\frac{1}{2})^4}=\frac{32}{275} $$
	
	$$ \gamma(z_{22}) = \frac{\pi_2 \cdot p(x_2|\mu_2)}{\pi_1 \cdot p(x_2|\mu_1)+\pi_2 \cdot p(x_2|\mu_2)} = \frac{\binom {5}{1} \cdot \frac{2}{3} \cdot (\frac{1}{2})^1 \cdot (\frac{1}{2})^4}{\binom {5}{1} \cdot \frac{1}{3} \cdot (\frac{2}{3})^1 \cdot (\frac{1}{3})^4+\binom {5}{1} \cdot \frac{2}{3} \cdot (\frac{1}{2})^1 \cdot (\frac{1}{2})^4}=\frac{243}{275} $$
	
	$$ \gamma(z_{31}) = \frac{\pi_1 \cdot p(x_3|\mu_1)}{\pi_1 \cdot p(x_3|\mu_1)+\pi_2 \cdot p(x_3|\mu_2)} = \frac{\binom {5}{2} \cdot \frac{1}{3} \cdot (\frac{2}{3})^2 \cdot (\frac{1}{3})^3}{\binom {5}{2} \cdot \frac{1}{3} \cdot (\frac{2}{3})^2 \cdot (\frac{1}{3})^3+\binom {5}{2} \cdot \frac{2}{3} \cdot (\frac{1}{2})^2 \cdot (\frac{1}{2})^3}=\frac{64}{307} $$
	
	$$ \gamma(z_{32}) = \frac{\pi_2 \cdot p(x_3|\mu_2)}{\pi_1 \cdot p(x_3|\mu_1)+\pi_2 \cdot p(x_3|\mu_2)} = \frac{\binom {5}{2} \cdot \frac{2}{3} \cdot (\frac{1}{2})^2 \cdot (\frac{1}{2})^3}{\binom {5}{2} \cdot \frac{1}{3} \cdot (\frac{2}{3})^2 \cdot (\frac{1}{3})^3+\binom {5}{2} \cdot \frac{2}{3} \cdot (\frac{1}{2})^2 \cdot (\frac{1}{2})^3}=\frac{243}{307} $$
	
	$$ \gamma(z_{41}) = \frac{\pi_1 \cdot p(x_4|\mu_1)}{\pi_1 \cdot p(x_4|\mu_1)+\pi_2 \cdot p(x_4|\mu_2)} = \frac{\binom {5}{3} \cdot \frac{1}{3} \cdot (\frac{2}{3})^3 \cdot (\frac{1}{3})^2}{\binom {5}{3} \cdot \frac{1}{3} \cdot (\frac{2}{3})^3 \cdot (\frac{1}{3})^2+\binom {5}{3} \cdot \frac{2}{3} \cdot (\frac{1}{2})^3 \cdot (\frac{1}{2})^2}=\frac{128}{371} $$
	
	$$ \gamma(z_{42}) = \frac{\pi_2 \cdot p(x_4|\mu_2)}{\pi_1 \cdot p(x_4|\mu_1)+\pi_2 \cdot p(x_4|\mu_2)} = \frac{\binom {5}{3} \cdot \frac{2}{3} \cdot (\frac{1}{2})^3 \cdot (\frac{1}{2})^2}{\binom {5}{3} \cdot \frac{1}{3} \cdot (\frac{2}{3})^3 \cdot (\frac{1}{3})^2+\binom {5}{3} \cdot \frac{2}{3} \cdot (\frac{1}{2})^3 \cdot (\frac{1}{2})^2}=\frac{243}{371} $$
	
	Therefore, we obtain the following table with responsibilities:
	
	\begin{table}[H]
		\centering
		\begin{tabular}{lrr}
			x       & $\gamma (z_{n_1})$          & $\gamma (z_{n_2})$          \\ \midrule
			(2, 3) & $\sfrac{64}{307}$ & $\sfrac{243}{307}$  \\ 
			\addlinespace
			(1, 4) & $\sfrac{32}{275}$ & $\sfrac{243}{275}$  \\
			\addlinespace
			(2, 3) & $\sfrac{64}{307}$ & $\sfrac{243}{307}$  \\
			\addlinespace
			(3, 2) & $\sfrac{128}{371}$ & $\sfrac{243}{371}$  \\
			\bottomrule
		\end{tabular}
	\end{table}
	
	
	\subsection*{Exercise 1.2}
	
	Now we are ready to update $\pi$ and $\mu$:
	
	$$\pi_k = \frac{N_k}{N}, \quad N_k = \sum\limits_{n=1}^N \gamma(z_{n_k})$$
	
	First, we compute $N_1$ and $N_2$:
	
	$$ N_1 = \frac{64}{307} + \frac{32}{275} + \frac{64}{307} + \frac{128}{371} \approx 0.878$$
	$$ N_2 = \frac{243}{307} + \frac{243}{275} + \frac{243}{307} + \frac{243}{371} \approx 3.122$$
	
	Now, we can calculate $\pi_1$ and $\pi_2$:
	$$\pi_1 = \frac{N_1}{N} = \frac{0.878}{4} \approx 0.219$$
	$$\pi_2 = \frac{N_2}{N} = \frac{3.122}{4} \approx 0.781$$
	
	In order to update $\mu_1$, we need to calculate the updated number of H's and T's for $\gamma(z_{n_1})$:
	
	$$N_{H1}= 2 \cdot \frac{64}{307} + 1 \cdot  \frac{32}{275}  + 2 \cdot \frac{64}{307} + 3 \cdot \frac{128}{371} \approx 1.985$$
	
	$$N_{T1}= 3 \cdot \frac{64}{307} + 4 \cdot  \frac{32}{275}  + 3 \cdot \frac{64}{307} + 2 \cdot \frac{128}{371} \approx 2.406$$
	
	Now, $\mu_1$ can be calculated as:
	$$\mu_1 = \frac{1.985}{1.985+2.406} \approx 0.452 $$
	
	
	Similarly, in order to update $\mu_2$, we need to calculate the updated number of H's and T's for $\gamma(Z_{n2})$:
	
	$$N_{H2}= 2 \cdot \frac{243}{307} + 1 \cdot  \frac{243}{275}  + 2 \cdot \frac{243}{307} + 3 \cdot \frac{243}{371} \approx 6.015$$
	
	$$N_{T1}= 3 \cdot \frac{243}{307} + 4 \cdot  \frac{243}{275}  + 3 \cdot \frac{243}{307} + 2 \cdot \frac{243}{371} \approx 9.594$$
	
	Now, $\mu_2$ can be calculated as:
	$$\mu_2 = \frac{6.015}{6.015+9.594} \approx 0.385 $$
	
	
	\section*{Part 2}
	\textbf{Important note}: we assume that latent variables are enumerated as $z_1$ and $z_2$ (not $z_0$ and $z_1$), therefore, we have states 1 and 2. It stays true until the end of this exercises and influences all the notations used in this exercise.
	
	We consider the simple two-state hidden Markov model M based on the tossing of two coins. The parameters
	of M are:
	\begin{enumerate}
		\item The initial probabilities: $\Pi = [0.4, 0.6]$.
		\item The transition probabilities 
		$$ A = \left(\begin{smallmatrix} 0.4 & 0.6 \\ \\ 0.7 & 0.3 \end{smallmatrix} \right)  $$
		So the transition probability from state 1 to state 2 is $ A(1, 2) $ which equals 0.6.
		\item The emission probabilities $\Phi_1 = [0.3, 0.7]$ , $\Phi_2 = [0.6, 0.4]$, first number is probability of
		head H.
	\end{enumerate}
	
	\subsection*{Exercise 2.1}
	
	
	
	
\end{document}